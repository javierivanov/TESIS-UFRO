
\leftskip=0cm 
\rightskip=0cm

\begin{center}
\begin{tabularx}{\textwidth}{p{3cm} >{\centering\arraybackslash}X}
	\vspace{0pt} 
	\includegraphics[width=2.8cm]{logo}
   	& 
   	\vspace{10pt} \textbf{ \uppercase{universidad de la frontera \linebreak facultad de ingeniería y ciencias\linebreak depto. cs. computación e informática}}
	
\end{tabularx}
\end{center}

\null
\vfill

\begin{center}
	\textbf{ \uppercase{
	``\titulotesis''
	} }
\end{center}

\null
\vfill

\begin{center}
	\textbf{ \uppercase{
	\nombre \\
	\anyo
	} }
\end{center}

\clearpage
\newpage\null\thispagestyle{empty}\newpage

\begin{center}
\begin{tabularx}{\textwidth}{p{3cm} >{\centering\arraybackslash}X}
	\vspace{0pt} 
	\includegraphics[width=2.8cm]{logo}
   	& 
   	\vspace{10pt} \textbf{ \uppercase{universidad de la frontera \linebreak facultad de ingeniería y ciencias\linebreak depto. cs. computación e informática}}
	
\end{tabularx}
\end{center}

\null
\vfill

\begin{center}
	\textbf{ \uppercase{
	``\titulotesis''
	} }
\end{center}

\null
\vfill

\rightskip=0cm
\noindent\rule{\textwidth}{1.5pt}
\setlength{\parindent}{0cm}
{
	\raggedright

	\uppercase{\textbf{trabajo para optar al título \\de ingeniero civil industrial mención informática}
}
\noindent\rule{\textwidth}{1.5pt}

\vspace{15 mm}
\leftskip=0cm

\uppercase{\textbf{\hfill Profesor guía:} \profesorguia}\\
\vspace*{0.5cm}
\uppercase{\textbf{\hfill Profesor co-guía:} \profesorcoguia}
}


\null
\vfill

\begin{center}
	\textbf{ \uppercase{\nombre \\ \anyo}}
\end{center}

\clearpage

\leftskip=0cm
\rightskip=0cm
\setlength{\parindent}{0cm}

\begin{center}
	\textbf{ \uppercase{
	``\titulotesis'' \\
	\vspace{5 mm}
	\nombre \\
	\vspace{5 mm}
	comisión examinadora
	} }
	\vspace{25 mm}

	\textbf{ \uppercase{\profesorguia} \\
		Profesor Guía
	}
	
	\vspace*{1.5cm}
		
	\textbf{ \uppercase{\profesorcoguia} \\
		Profesor Co-Guía
	}

\end{center}

\vspace{4cm}

\begin{tabularx}{\textwidth}{ X X X } 
	Dr. CARLOS FERNANDO CARES GALLARDO & Mg. JORGE ALBERTO HOCHSTETTER DIEZ  & Otro profesor connotado\\
	\multicolumn{1}{c}{Profesor Examinador 1} &	\multicolumn{1}{c}{Profesor Examinador 2} & \multicolumn{1}{c}{Profesor Examinador 3} \\					
\end {tabularx}
\vfill
\begin{flushright}
	\textbf{Nota trabajo escrito :} \tab{} \\ 	
	\textbf{Nota examen :} \tab{} \\ 
	\textbf{Nota final :} \tab{} \\ 		
\end{flushright}

