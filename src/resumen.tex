

{
\fontsize{16pt}{19.2pt}%
\selectfont%
\center%
\begingroup%

%\uppercase{
%dedicatoria 
%}
\endgroup
\endcenter
}

{
\topskip0pt
\vspace*{\fill}
\hfill \textit{dedicatoria}
\vspace*{\fill}
}

\clearpage

\begin{comment}
{
\fontsize{16pt}{19.2pt}%
\selectfont%
\center%
\begingroup%

\uppercase{
	agradecimientos 
}
	

\endgroup
\endcenter
}


\clearpage

\end{comment}

{
\fontsize{16pt}{19.2pt}%
\selectfont%
\center%
\begingroup%

\uppercase{
	resumen 
}	

\endgroup
\endcenter
}

\setlength{\parindent}{1cm}
\setlength{\parskip}{5pt}

%%%%%
El presente informe muestra el desarrollo del trabajo de título “Etapa de pre-procesamiento de minería de datos para registros de ACS”.
El Ataca Large Millimiter/ Submillimeter Array (ALMA) es actualmente el observatorio astronómico más grande del mundo y como tal tiene un grado de  complejidad muy alto. Los sistemas de ALMA son controlados por un software llamado Alma Common Software (ACS), sistema que se encarga de todas las operaciones relacionadas a la observación. 
Este trabajo se enfoca específicamente en el proceso de observación, en el cual, es de interés poder monitorear y analizar su comportamiento en base a los registros generados por ACS. 
Este trabajó consistió en la producción de una herramienta capaz de interpretar los regisrtos de ACS e identificar los diferentes estados observables dentro de los procesos.
Por otra parte también se generó una herramienta para descubrir secuencias en la ejecución de las observaciones, estas secuencias permiten analizar.
